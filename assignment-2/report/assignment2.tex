% This is a sample LaTeX input file.  (Version of 12 August 2004.)
%
% A '%' character causes TeX to ignore all remaining text on the line,
% and is used for comments like this one.

\documentclass{article}      % Specifies the document class

                             % The preamble begins here.
\title{\bf Principles of Computer Systems Design\\ {\Large Assignment 2}}  % Declares the document's title.
\author{Tudor Dragan\\
Gabriel Carp}      % Declares the author's name.
\date{December 2, 2014}      % Deleting this command produces today's date.

\usepackage{verbatimbox}
\usepackage{listings}
\usepackage{color}
\usepackage[]{amsmath}
\usepackage[english]{babel}
\usepackage[utf8]{inputenc}
\usepackage{graphicx}
\usepackage{moreverb}
\usepackage{hyperref}
\usepackage[T1]{fontenc} % font
\usepackage{program}
\usepackage[top=1.5in, bottom=1.5in, left=1.4in, right=1.4in]{geometry}
\usepackage[super]{nth}

\definecolor{dkgreen}{rgb}{0,0.6,0}
\definecolor{gray}{rgb}{0.5,0.5,0.5}
\definecolor{mauve}{rgb}{0.58,0,0.82}

\lstset{frame=tb,
      language=Java,
      aboveskip=3mm,
      belowskip=3mm,
      showstringspaces=false,
      columns=flexible,
      basicstyle={\small\ttfamily},
      numbers=none,
      numberstyle=\tiny\color{gray},
      keywordstyle=\color{blue},
      commentstyle=\color{dkgreen},
      stringstyle=\color{mauve},
      breakatwhitespace=true
      tabsize=3
}
\newcommand{\ip}[2]{(#1, #2)}
                             % Defines \ip{arg1}{arg2} to mean
                             % (arg1, arg2).

%\newcommand{\ip}[2]{\langle #1 | #2\rangle}
                             % This is an alternative definition of
                             % \ip that is commented out.

\begin{document}             % End of preamble and beginning of text.

\maketitle                   % Produces the title.

\section*{Question 1: Serializability \& Locking} 

Conflict-serializability is defined by equivalence to a serial schedule (no overlapping transactions) with the same transactions, such that both schedules have the same sets of respective chronologically ordered pairs of conflicting operations (same precedence relations of respective conflicting operations).

A schedule is conflict-serializable if and only if it's precedence graph has no cycles. This is a graph of nodes and vertices, where the nodes are the transaction names and the vertices are attribute collisions.

\subsection*{Schedule 1}

Schedule 1 \emph{is not} conflict serializable because the graph is \emph{cyclic}:
\begin{itemize}
\item $T_1$ - $T_2$: read-write conflict on $X$.
\item $T_2$ - $T_3$: write-read conflict on $Z$.
\item $T_3$ - $T_1$: read-write conflict on $Y$.
\end{itemize}

\subsection*{Schedule 2}

Schedule 2 \emph{is} conflict serializable because the precedence graph is \emph{acyclic}:
\begin{itemize}
\item $T_1$ - $T_2$: $X$ is accessed by $T_2$ after $T_1$ has committed.
\item $T_2$: $Y$ is only accessed by $T_2$.
\item $T_3$ - $T_2$: $Z$ exclusively locked by $T_3$ is released prior to $T_2$ acquiring a
shared lock.
\end{itemize}

\section*{Question 2: Optimistic Concurrency Control}

\subsection*{Scenario 1}

Because $T_1$ finishes before $T_3$ starts, the \nth{1} condition holds. We have to check that the \nth{2} condition holds for $T_2$ and $T_3$, but because $T_2$ writes the object that $T_3$ reads from, this condition does not hold. Therefore $T_3$ hast to \emph{rollback}.

\subsection*{Scenario 2}

Because $T_1$ finishes before $T_3$ starts, the \nth{1} condition holds. We have to check that the \nth{2} condition holds for $T_2$ and $T_3$, but because $T_2$ writes the object that $T_3$ reads from, this condition does not hold. Therefore $T_3$ hast to \emph{rollback}.


\section*{Programming Task}


\end{document}               % End of document.