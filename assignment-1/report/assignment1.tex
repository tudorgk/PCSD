% This is a sample LaTeX input file.  (Version of 12 August 2004.)
%
% A '%' character causes TeX to ignore all remaining text on the line,
% and is used for comments like this one.

\documentclass{article}      % Specifies the document class

                             % The preamble begins here.
\title{\bf Principles of Computer Systems Design\\ {\Large Assignment 1}}  % Declares the document's title.
\author{Tudor Dragan\\
Gabriel Carp}      % Declares the author's name.
\date{December 25, 2014}      % Deleting this command produces today's date.

\usepackage{verbatimbox}
\usepackage{listings}
\usepackage{color}
\usepackage[]{amsmath}
\usepackage[english]{babel}
\usepackage[utf8]{inputenc}
\usepackage{graphicx}
\usepackage{moreverb}
\usepackage{hyperref}
\usepackage[T1]{fontenc} % font
\usepackage{program}
\usepackage[top=1.5in, bottom=1.5in, left=1.4in, right=1.4in]{geometry}

\definecolor{dkgreen}{rgb}{0,0.6,0}
\definecolor{gray}{rgb}{0.5,0.5,0.5}
\definecolor{mauve}{rgb}{0.58,0,0.82}

\lstset{frame=tb,
      language=Java,
      aboveskip=3mm,
      belowskip=3mm,
      showstringspaces=false,
      columns=flexible,
      basicstyle={\small\ttfamily},
      numbers=none,
      numberstyle=\tiny\color{gray},
      keywordstyle=\color{blue},
      commentstyle=\color{dkgreen},
      stringstyle=\color{mauve},
      breakatwhitespace=true
      tabsize=3
}
\newcommand{\ip}[2]{(#1, #2)}
                             % Defines \ip{arg1}{arg2} to mean
                             % (arg1, arg2).

%\newcommand{\ip}[2]{\langle #1 | #2\rangle}
                             % This is an alternative definition of
                             % \ip that is commented out.

\begin{document}             % End of preamble and beginning of text.

\maketitle                   % Produces the title.

\section{Question 1: Fundamental Abstractionst}      % Produces section heading.  Lower-level
                             % sections are begun with similar 
                             % \subsection and \subsubsection commands.

A simple solution would be to lay out the machines in one single address space. If we assume that all the machines have the exact same memory size and they are perfectly fault tolerant, a simple arithmetic calculation can be used in the translation of the addresses for each machine. We could find the machine with \emph{address {\bf mod} machineMemorySize } and the find the local address by subtracting the \emph{offSet}.\\

Because we are not working with an ideal scenario, we must translate the address by doing a look-up for finding the machine that contains the page\footnote{A page, memory page, or virtual page is a fixed-length contiguous block of virtual memory, described by a single entry in the page table.} in memory. The look-up should be either a mapping or a function that retrieves the \emph{machine number} and \emph{local address} of the page. By using this solution the memory can be spread out in different sizes.\\

In order to do these translations we would need to implement a centralized mapping system (a system similar to DNS for websites). In our case we use the \textbf{Central Server Algorithm} where a central-server maintains all the shared data. It translates the addresses and services the read requests from other nodes or clients by returning the data items to them. It updates the data on write requests by clients and returns acknowledgment messages. A timeout can be employed to resend the requests in case of failed acknowledgments. Duplicate write requests can be detected by associating sequence numbers with write requests. A failure condition is returned to the application trying to access shared data after several retransmissions without a response.\\

Although, the central-server algorithm is simple to implement, the central-server can become a \emph{bottleneck}. To overcome this problem, shared data can be distributed among several servers. In such a case, clients must be able to locate the appropriate server for every data access. Multicasting data access requests is undesirable as it does not reduce the load at the servers compared to the central-server scheme. A better way to distribute data is to partition the shared data by address and use a mapping function to locate the appropriate server\footnote{\url{http://cs.gmu.edu/cne/modules/dsm/blue/ctr_ser_Al.html}}.\\

The API contains the address translations and the READ and WRITE functions that we assume as being atomic functions\footnote{In concurrent programming, an operation (or set of operations) is atomic, linearizable, indivisible or uninterruptible if it appears to the rest of the system to occur instantaneously. Atomicity is a guarantee of isolation from concurrent processes.}. The READ and WRITE functions need to translate the address by calculating the address offset and machine identifier that has the page.\\

\begin{figure}[htbp]
\begin{center}
\begin{lstlisting}
private {machineID, localOffset} translateAddress(address) {
	{machineID, offset} = lookup(address); // the offset represents the first address of the machine
	localOffset = address - offset; // get the local offset in the machine's memory  
	return {machineID,localOffset};  
	}
\end{lstlisting}
\caption{Translate function}
\label{Translate function}
\end{center}
\end{figure}



Because printing is different from typewriting,
there are a number of things that you have to do
differently when preparing an input file than if
you were just typing the document directly.
Quotation marks like
       ``this'' 
have to be handled specially, as do quotes within
quotes:
       ``\,`this'            % \, separates the double and single quote.
        is what I just 
        wrote, not  `that'\,''.  

Dashes come in three sizes: an 
       intra-word 
dash, a medium dash for number ranges like 
       1--2, 
and a punctuation 
       dash---like 
this.

A sentence-ending space should be larger than the
space between words within a sentence.  You
sometimes have to type special commands in
conjunction with punctuation characters to get
this right, as in the following sentence.
       Gnats, gnus, etc.\ all  % `\ ' makes an inter-word space.
       begin with G\@.         % \@ marks end-of-sentence punctuation.
You should check the spaces after periods when
reading your output to make sure you haven't
forgotten any special cases.  Generating an
ellipsis
       \ldots\               % `\ ' is needed after `\ldots' because TeX 
                             % ignores spaces after command names like \ldots 
                             % made from \ + letters.
                             %
                             % Note how a `%' character causes TeX to ignore 
                             % the end of the input line, so these blank lines 
                             % do not start a new paragraph.
                             %
with the right spacing around the periods requires
a special command.

\LaTeX\ interprets some common characters as
commands, so you must type special commands to
generate them.  These characters include the
following:
       \$ \& \% \# \{ and \}.

In printing, text is usually emphasized with an
       \emph{italic}  
type style.  

\begin{em}
   A long segment of text can also be emphasized 
   in this way.  Text within such a segment can be 
   given \emph{additional} emphasis.
\end{em}

It is sometimes necessary to prevent \LaTeX\ from
breaking a line where it might otherwise do so.
This may be at a space, as between the ``Mr.''\ and
``Jones'' in
       ``Mr.~Jones'',        % ~ produces an unbreakable interword space.
or within a word---especially when the word is a
symbol like
       \mbox{\emph{itemnum}} 
that makes little sense when hyphenated across
lines.

Footnotes\footnote{This is an example of a footnote.}
pose no problem.

\LaTeX\ is good at typesetting mathematical formulas
like
       \( x-3y + z = 7 \) 
or
       \( a_{1} > x^{2n} + y^{2n} > x' \)
or  
       \( \ip{A}{B} = \sum_{i} a_{i} b_{i} \).
The spaces you type in a formula are 
ignored.  Remember that a letter like
       $x$                   % $ ... $  and  \( ... \)  are equivalent
is a formula when it denotes a mathematical
symbol, and it should be typed as one.

\section{Displayed Text}

Text is displayed by indenting it from the left
margin.  Quotations are commonly displayed.  There
are short quotations
\begin{quote}
   This is a short quotation.  It consists of a 
   single paragraph of text.  See how it is formatted.
\end{quote}
and longer ones.
\begin{quotation}
   This is a longer quotation.  It consists of two
   paragraphs of text, neither of which are
   particularly interesting.

   This is the second paragraph of the quotation.  It
   is just as dull as the first paragraph.
\end{quotation}
Another frequently-displayed structure is a list.
The following is an example of an \emph{itemized}
list.
\begin{itemize}
   \item This is the first item of an itemized list.
         Each item in the list is marked with a ``tick''.
         You don't have to worry about what kind of tick
         mark is used.

   \item This is the second item of the list.  It
         contains another list nested inside it.  The inner
         list is an \emph{enumerated} list.
         \begin{enumerate}
            \item This is the first item of an enumerated 
                  list that is nested within the itemized list.

            \item This is the second item of the inner list.  
                  \LaTeX\ allows you to nest lists deeper than 
                  you really should.
         \end{enumerate}
         This is the rest of the second item of the outer
         list.  It is no more interesting than any other
         part of the item.
   \item This is the third item of the list.
\end{itemize}
You can even display poetry.
\begin{verse}
   There is an environment 
    for verse \\             % The \\ command separates lines
   Whose features some poets % within a stanza.
   will curse.   

                             % One or more blank lines separate stanzas.

   For instead of making\\
   Them do \emph{all} line breaking, \\
   It allows them to put too many words on a line when they'd rather be 
   forced to be terse.
\end{verse}

Mathematical formulas may also be displayed.  A
displayed formula 
is 
one-line long; multiline
formulas require special formatting instructions.
   \[  \ip{\Gamma}{\psi'} = x'' + y^{2} + z_{i}^{n}\]
Don't start a paragraph with a displayed equation,
nor make one a paragraph by itself.

\end{document}               % End of document.