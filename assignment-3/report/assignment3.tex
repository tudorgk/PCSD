% This is a sample LaTeX input file.  (Version of 12 August 2004.)
%
% A '%' character causes TeX to ignore all remaining text on the line,
% and is used for comments like this one.

\documentclass{article}      % Specifies the document class

                             % The preamble begins here.
\title{\bf Principles of Computer Systems Design\\ {\Large Assignment 2}}  % Declares the document's title.
\author{Tudor Dragan\\
Gabriel Carp\\
Sokratis Siozos - Drosos}      % Declares the author's name.
\date{December 9, 2014}      % Deleting this command produces today's date.

\usepackage{verbatimbox}
\usepackage{listings}
\usepackage{color}
\usepackage[]{amsmath}
\usepackage[english]{babel}
\usepackage[utf8]{inputenc}
\usepackage{graphicx}
\usepackage{moreverb}
\usepackage{hyperref}
\usepackage[T1]{fontenc} % font
\usepackage{program}
\usepackage[top=1.5in, bottom=1.5in, left=1.4in, right=1.4in]{geometry}
\usepackage[super]{nth}

\definecolor{dkgreen}{rgb}{0,0.6,0}
\definecolor{gray}{rgb}{0.5,0.5,0.5}
\definecolor{mauve}{rgb}{0.58,0,0.82}

\lstset{frame=tb,
      language=Java,
      aboveskip=3mm,
      belowskip=3mm,
      showstringspaces=false,
      columns=flexible,
      basicstyle={\small\ttfamily},
      numbers=none,
      numberstyle=\tiny\color{gray},
      keywordstyle=\color{blue},
      commentstyle=\color{dkgreen},
      stringstyle=\color{mauve},
      breakatwhitespace=true
      tabsize=3
}
\newcommand{\ip}[2]{(#1, #2)}
                             % Defines \ip{arg1}{arg2} to mean
                             % (arg1, arg2).

%\newcommand{\ip}[2]{\langle #1 | #2\rangle}
                             % This is an alternative definition of
                             % \ip that is commented out.

\begin{document}             % End of preamble and beginning of text.

\maketitle                   % Produces the title.

\section*{Question 1: Recovery Concepts} 


\subsubsection*{1.In a system implementing force and no-steal, is it necessary to implement a scheme for redo? What about a scheme for undo? Explain why.}

In a system like this, we do not have to implement a scheme for redo because the force approach is used, which according to the theory means that we do not have to redo the changes of a committed transaction if there is a subsequent crash. As a result, all these changes are guaranteed to have been written to disk at the time the data has been commited. \\

If a no-steal approach is used, we also do not have to undo the changes of an aborted transaction (because these changes have not been written to disk),\\


\subsubsection*{2.  What is the difference between nonvolatile and stable storage? What types of failures are survived by each type of storage?}

According to the theory, a volatile memory is one whose mechanism of retaining information consumes energy. If its power supply is interrupted for some reason, it forgets its information content. Such a memory is RAM. On the other hand a stable or non-volatile storage retains its content even if the power supply is interrupted. When power is again available, READ operations return the same values as before.\\

\subsubsection*{3.  In a system that implements Write-Ahead Logging, which are the two situations in which the log tail must be forced to stable storage? Explain why log forces are necessary in these situations and argue why they are sufficient for durability.}
A situation in which the log tail is forced to stable storage, is when a transaction is committed, even if a no-force approach is being used.
TODO TODO TODO\\

\section*{Question 2: ARIES}

\begin{table}[h]
\begin{center}
\begin{tabular}{|c|c|c|c|c|}
\hline
\textbf{LSN} & \textbf{LAST\_LSN} & \textbf{TRAN\_ID} & \textbf{TYPE} & \textbf{PAGE\_ID} \\ \hline
1            & -                  & -                 & begin CKPT    & -                 \\ 
2            & -                  & -                 & end CKPT      & -                 \\ 
3            & NULL               & T1                & update        & P2                \\ 
4            & 3                  & T1                & update        & P1                \\ 
5            & NULL               & T2                & update        & P5                \\
6            & NULL               & T3                & update        & P3                \\
7            & 6                  & T3                & commit        & -                 \\ 
8            & 5                  & T2                & update        & P5                \\
9            & 8                  & T2                & update        & P3                \\ 
10           & 6                  & T3                & end           &                   \\ \hline
\end{tabular}
\caption{The LOG before the crash}
\label{The LOG before the crash}
\end{center}
\end{table}

\begin{enumerate}
\item %question 1 

To create the \emph{Transaction} and \emph{Dirty Page} tables we need to go through the \emph{Analysis Phase}. The algorithm states that we must reconstruct these tables after the last \emph{end CKPT} mark. We scan the log forward from \emph{LSN 3} and add the transactions in the \emph{Transaction Table} that have not encountered an \emph{end} record. We update the transaction's  \emph{last LSN} entry by updating it to the current LSN of the transaction. The \emph{Transaction table} after the analysis phase would look like this: \\

%transaction table
\begin{table}[h]
\begin{center}
\begin{tabular}{|c|c|c|}
\hline
\textbf{Tx ID} & \textbf{Status} & \textbf{LastLSN} \\ \hline
T1             & active          & 4                \\ 
T2             & active          & 9                \\ \hline
\end{tabular}
\caption{Transaction Table}
\label{Transaction Table}
\end{center}
\end{table}

For the \emph{Dirty Page table}, we need to check the \emph{update} records in the log. We need to store the earliest LSN for that page modification. If the a page is modified by a transaction that has successfully ended it's commit action, we remove it from the \emph{Dirty Page table}. The \emph{Dirty Page table} is presented below:\\

%dirty page table
\begin{table}[h]
\begin{center}
\begin{tabular}{|c|c|}
\hline
\textbf{Page ID} & \textbf{recLSN} \\ 
\hline
P2               & 3               \\
P1               & 4               \\
P5               & 5               \\ 
\hline
\end{tabular}
\caption{Dirty Page Table}
\label{Dirty Page Table}
\end{center}
\end{table}

\item %question 2
The transactions that have not committed, are considered \emph{losing transactions} because they have not been written to the non-volatile storage. In our case, the only transaction that completed it's commit is $T3$. Therefore, the loser transactions are $T1$ and $T2$.\\

\item %question 3
The REDO phase starts from the smallest \emph{recLSN} in dirty page table after the Analysis phase. In our case the smallest LSN is \emph{3}. The UNDO phase ends at the oldest log record of an active transaction at crash. In our case the minimum LSN is \emph{3}.\\

\item %question 4
Because in the REDO phase we actually reapply everything, all the pages that were updated after the checkpoint will be modified. The set should therefore be:
\begin{verbatim}
RedoPageSet = {P2,P1,P5,P3}
\end{verbatim}

\item %question 5
In the UNDO stage we need to rollback all the transactions that haven't committed their changes. The UNDO set is composed of the LSN's that are attributed to a loser transaction. In our case the UNDO log record set will be:
\begin{verbatim}
UndoLogRecords = {9,8,5,4,3}
\end{verbatim}

\item %question 6
When building the log after a crash we must rollback the transactions that haven't committed (loser transactions). Thus, we must add CLR records to the log to undo the modifications that a loser transaction has made. the log table after the recovery procedure will look like this:

\begin{table}[h]
\begin{center}
\begin{tabular}{|c|c|}
\hline
\textbf{LSN} & \textbf{LOG}         \\ \hline
1            & begin CKPT           \\ \hline
2            & end CKPT             \\ \hline
3            & update: T1 writes P2 \\ \hline
4            & update: T1 writes P1 \\ \hline
5            & update: T2 writes P5 \\ \hline
6            & update: T3 writes P3 \\ \hline
7            & T3 commit            \\ \hline
8            & update: T2 writes P5 \\ \hline
9            & update: T2 writes P3 \\ \hline
10           & T3 end               \\ \hline
\textbf{X}   & \textbf{CRASH}       \\ \hline
11           & CLR: Undo T2 LSN 9   \\ \hline
12           & CLR: Undo T2 LSN 8   \\ \hline
13           & CLR: Undo T2 LSN 5   \\ \hline
14           & T2 End               \\ \hline
15           & CLR: Undo T1 LSN 4   \\ \hline
16           & CLR: Undo T1 LSN 3   \\ \hline
17           & T1 End               \\ \hline
\end{tabular}
\caption{Log after Recovery}
\label{Log after Recovery}
\end{center}
\end{table}
\end{enumerate}
To add the CLR records we must keep track of the last LSN's of the loser transactions. We choose the largest LSN ( in out case 9) and we observe that the record is an \emph{update}. We then undo the update and write a CLR (Compensation Log Record) to record that we did the rollback. Afterwards we update the CLR's prevLSN with the last LSN of the transaction (in our example here, we set the prevLSN to 8). We apply this algorithm until we have no more loser transaction updates to undo. \\

\section*{Programming Task}

\subsection*{Testing}



\subsection*{Questions for Discussion on the Performance Measure-
ments}

\subsubsection*{ Discuss in detail the setup you have created for your experiments. In particular, document your data generation procedures, hardware employed, measurement procedures (e.g., number of repetitions, statistics used such as average or deviation), and any other considerations you made. In the evaluation of this question, we will consider not only your thoroughness, but also whether you provide a brief justification/rationale for your decisions.}



\subsubsection*{ Show and explain the plots for throughput and latency that you obtained. As described above, we expect two plots: one for throughput and one for latency. Each plot should include two curves: one for executions in the same address space, and one for executions across address spaces. Describe the trends observed and any other effects. Explain why you observe these trends and how much that matches your Expectations. }


\subsubsection*{How reliable are the metrics and the workloads for predicting the performance of the bookstore? Are the metrics well chosen? What additional metrics would you choose to demonstrate the performance of the bookstore?}


\end{document}               % End of document.